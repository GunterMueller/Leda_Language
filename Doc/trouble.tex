\chapter{Unresolved Language Issues in Leda}\label{w0}

Although the Leda language as described in this manual is now reasonably
mature, there remain a few issues that are frequently
sources of error or frustration, and are likely targets for future changes
in the language (or in the next generation multiparadigm language).
In this chapter we will describe the most critical of these issues.
(In addition, the current Leda implementation makes a number of limiting
assumptions, thus supporting only a subset of the language defined
in Chapter~\ref{gr0}.
This, however, is simply an issue of implementation, and not language
design.)

In reading the following, keep in mind that an overriding design
objective has been to keep the language as simple as possible, but no
simpler.  Many issues could be addressed by introducing new {\em mechanisms},
and one must always weight the cost (both conceptual and practical) against
the benefits.

\section{Semantics of Assignment and Parameter Passing}

In several places we have
described the particular semantics of assignment and parameter passing
in Leda.  There seems little chance that the semantics of assignment
will be changed, but there is some possibility that the technique used
in parameter passing can be modified, particularly if a mechanism
for cloning values is introduced (see Section~\ref{w1}).
The principle problem with current parameter passing mechanism is that
it does not prevent pass-by-value parameters from being modified from within
a called procedure.  There have been various proposals made to address this
difficulty:

\begin{itemize}
\item
Make a clone or copy of a value before passing as a by-value parameter.
(Requires a general cloning facility).
\item
Make by-value parameters constant, and forbid operations on constant
values.  (Currently, operations on constant values are not forbidden,
only assignment to constants).  This seems too restrictive, as many
existing programs would then cease to work.
\item
Somehow characterize which functions are constant preserving and
which are not, and only allow constant preserving functions to
be performed on by-value parameters.
It is not at all clear, however, how to discover which functions
preserve constant values, and which do not.
\end{itemize}

\section{Modularization Facilities}

Currently a Leda program is simply one large file.  There are ``include''
facilities, of course, so that data structures can be moved from one
application to another, but no mechanism for separate compilation.

\section{Restricted Scoping}

Currently all variables declared within a class scope are accessible
using the dot operator.  Good style dictates that such values
should never be modified outside the class functions, but the
language does not enforce this restriction.  A proposed change
would be to include C++ style modifiers for declarations which would
separate those values which were purely local to a class from those
which are permitted to be modified outside the class.

\section{Shared and Overridden Data Fields}

In an earlier version of Leda it was possible to declare data fields which
were shared by all instances of a class.  It was also possible to
override data fields declared in parent classes, in much the same
way that methods are overridden.  Both features were little used,
and due to difficulties in syntax or implementation, both have been
dropped from the current version of the language.

\section{The Subscript operator}

It is conventional in programming languages, going back to the days of
Fortran and ALGOL-60, to use braces of some sort to indicate
array subscripts.  Should Leda follow this tradition?

\begin{itemize}
\item
Arrays are currently only weakly part of the ``language'', if by this
one means the syntax recognized by the parser.  Instead, arrays are
simply a data structure that happens to be implemented in the standard
library.  (This is not entirely true, since array literals require
a special syntax, and thus the parser does need to know about the
``Array'' data type in order to give a type to these literals).
\item
Is it right to complicate the syntax of the language (by adding array
expressions) for just a single data structure?  Of course, one could
say that this is not really just for arrays -- onE could add productions
like:
\begin{cprog}

basicExpression:
	basicExpression LEFTBRACKET expression-list RIGHTBRACKET

\end{cprog}\noindent
and say this is simply a shorthand for the function call
\begin{cprog}

	expression.subscript(arguments);

\end{cprog}
where the number of expressions must match the number of arguments provided
for whatever the subscript operator is found in the class of the expression.
(This would permit users to define their own array-like things, such
as matrices or dictionaries).
\item
Presuming one did do something like the previous, what should the return
type be for the subscript operator?  Note that the semantics of
subscripts are different depending upon whether they appear as the
target of an assignment or as an expression.
There seems to be two choices -- either have separate operators
for these two operations (similar to the ``at'' and ``atPut'' functions
now in use in Leda, admittedly copied from Smalltalk), or introduce
some mechanism for explicitly
dealing with references as values (as in C++).  Neither of these seems
particularly appealing.
\end{itemize}

\section{Overloaded operators and/or functions}

Currently there is only a minimal amount of overloading of function
names permitted in Leda.  Multiple copies of operator
definitions (plus, times, ``and'' and ``or'' and the like) are permitted
at the global scope, and nowhere else.  Types of overloading that
are {\em not} permitted include:
\begin{itemize}
\item
overloading of other function names at the global scope
\item
overloading of operators in local scopes (either within classes or within
functions)
\item
overloading of other function names in local scopes
\end{itemize}

The argument for outlawing each of these is the same.  That is, that
currently if {\tt bar} is a function in scope {\tt foo}, the expression
{\tt foo.bar} has a single well defined and unambiguous meaning.
This is a value of type function (technically, a closure), and can
be assigned to a variable, passed as an argument, and so on.  In fact,
all procedure calls, such as:
\begin{cprog}

	foo.bar(x, y, z)

\end{cprog}\noindent
are actually treated in two steps.  First, the expression foo.bar is
evaluated, resulting in a closure value.  Then this closure is
invoked as a function.
Introducing any of the overloadings that are currently not permitted
would destroy this property.

(It is sometimes argued that at least operators should be permitted to
be overloaded in class scopes, since operators are never invoked without
arguments, but this is not true; several places in the book I refer
to expression such as ``integer.plus'' and expect it to mean the
binary function named plus defined in the class integer.  Allowing
overloading would make such an expression ambiguous.)

\section{Nested Classes}

This is probably less an open point, and more simply a bug in the
current language implementation.  A glance at the current Leda
grammar will indicate that I had originally intended for classes
and functions to be permitted to be nested arbitrarily -- functions
within functions, functions within classes, classes within functions,
or classes within classes.  While pondering the question of forward
references (see next section) the following awkward example was discovered: 

\begin{cprog}

class A;

	function foo();
	begin
		...
	end;

	class B of A;
	 ...
	end;

	function bar();
	begin
		...
	end;
end;

\end{cprog}

That is, within class A a new class B is defined, which subclasses from A.
Now, currently the Leda language has been carefully crafted so that it
can be parsed and compiled in one pass.  This being the case, do
instances of B inherit the function foo from class A?  what about
the function bar?  If an instance of B is assigned to a variable
declared as maintaining an A, can we perform the bar function on
the result?  There are a number of technical issues that complicate
this (meaning, subclasses are basically an extension of the parent
class, and one can't extend something that hasn't been completely
constructed yet).  Probably the most logical thing is to rule out
completely the type of construct shown here.

\section{Forward References for Classes or Functions}

In order versions of the Leda language I separated class definition
from method definition.  In the latest revision I have used the
BETA idea (in fact this actually goes all the way back to SIMULA) of just
putting everything together into one big structure.
While this greatly simplified the language description, it did have
the disadvantage that it makes it
difficult to write mutually recursive classes.

Consider first the analogy to mutually recursive functions.
By mutually recursive functions I mean that a function
A can call function B, which is some cases can call function A again.
Mutually recursive functions can sometimes (but not always) be handed
by nesting, as in
\begin{cprog}

	function A();

		function B();
		begin
			...
			A();
		end;

	begin
		...
		B();
		...
	end;	

\end{cprog}
It was the consideration of the analogy to classes that led me to discover
the problem with nested classes described in the previous section.

Another common way to get around the problem of mutually recursive
functions, when nesting doesn't work, is to create a variable of
type function, which is later defined:

\begin{cprog}

	var
		A : function();

	function B();
	begin
		...
		A();
		...
	end;

	function C();
	begin
		...
		B();
		...
	end;

	begin
		A := C;
	end;

\end{cprog}
This is a somewhat primitive facility that nevertheless permits functions
to be forward referenced (that is, named before they are defined).
A more general forward reference mechanism for functions could probably
be defined and easily implemented, but I don't know if the payoff is
worth the effort, because the situation is not that common,
either nesting or the variable trick is usually sufficient, and in
any case neither gives us any hint as to how to handle forward
referencing of mutually recursive classes.

The equivalent to the mutually recursive functions in classes is
a class A which in one of its methods creates and manipulates
an instance of class B,
and a class B in which in one of its methods creates and manipulates
an instance of class A.  In the old Leda (and in C++, Object Pascal, and
other languages which separate the class definition from the method
definitions) the solution is clear.  Give the class definition for A,
followed by the class definition for B, followed by the methods for A,
followed by the methods for B.  By the time the methods are seen
both class definitions have been read, so both are known.

\section{Copies or Clones}\label{w1}

A common operation is to make a copy, or clone, of a value.
Of course, it is easy to define a clone method in any {\em particular} class
to do this operation.  What is harder is to try to develop a system-wide
function or method to generalize this.

The most obvious approach would be to have a function defined in class
object, which would thus be inherited by all values.  Classes which
wanted to change the meaning of this (such as integers or arrays)
could then easily override the method.  But the difficulty with this
is the question -- what should be the return type associated with this
function?

A similar situation occurs with regards to the problem of ``reverse
polymorphism''.  That is, taking a value known to be of some superclass,
and trying to discover whether in fact it is an element of a subclass.
In the current Leda this is accomplished in a somewhat clumsy fashion,
through a parameterized function named {\tt typeTest} which is defined
as follows:

\begin{cprog}

function typeTest [T : object] (val : object, aClass : Class)->T;
begin
	if aClass.isInstance(val) then
		return cfunction Leda_object_cast(val)->T;
	return NIL;
end;

\end{cprog}

What is clumsy about this is that the class name must be both passed
as a type-parameter (which is used by the parser, not during execution)
and as a value parameter (used by the execution of the procedure).
But this seems to be the cost for having the mechanism defined
entirely in Leda, and not handed as a special case by the parser.

But this also means that subclasses can not define their own version of
``typeTest'', as would seemingly be necessary to make an equivalent copy
procedure.

\section{Enumerated Datatypes}

Older versions of the Leda language had enumerated datatypes -- differing
from their PASCAL or C cousins only in that they printed using their
symbolic form and could be iterated over using a relation.
But the amount of work needed by the parser and the changes to the
grammar and the run-time system necessitated by the introduction of
these features seems all out of proportion to their utility.

It also seemed to be the case that enumerated datatypes in previous
versions of the language were primitive, and not easily extensible (by
adding new functionality, for example).

In the latest Leda I have found a literal array of strings to be
almost as easy to use, and does not require any new language features.

(A half-way measure might be to introduce symbols, as in Lisp or Smalltalk.
The equivalent to an enumerated datatype would then be a literal array
of symbols, rather than strings).

\section{User Defined Constructors}

In my previous extensive work in Smalltalk and C++ I found that the large
majority of time (perhaps as high as 75\%) that constructors
were used, they simply initialized the data fields in the newly
generated object.  This is why in Leda this is the default behavior,
and doesn't require any additional effort on the part of the programmer.
Nevertheless, in the remaining 25\% of the time it would be nice
to have more general constructor facilities that could perform
actions other than simply initialization.

I have toyed with a couple of idea.  The BETA-like solution would be
to make classes more like functions, as in
\begin{cprog}

	class A(a, b, c : x) of B(a, b);
		...
	begin
		{ do initialization here }
	end;

\end{cprog}

This makes the class heading somewhat complicated to look at, (and even
harder to parse -- for example the parent field is now an expression,
not a type) but is a nice general mechanism.
Unfortunately, it limits one to a single constructor for every class.

Another possible solution is to permit some sort of special ``functions''
to appear within a class, such as is done in C++, where a function
with the same name as the class is treated differently from other
functions.  This allows multiple constructors for a single class,
but seems in other respects somewhat less elegant.

\section{Metaclasses}

Are classes objects or not?  If the answer is Yes, then what is their type?
This is commonly called metaclass programming, and is somewhat popular at
present.  I've always had a slight suspicion of metaclass programming,
thinking that the mechanism is far too complex for the benefits involved.
Leda does have class values -- see the description of class object
and class Class in the standard library.
Every object has a field of type Class,
which is holding the name of the class, its size, the pointer to its
parent class, and methods.  (The
latter don't actually appear in the class description).

Currently only two things are done with this information.
The first is the display
of the class name when printing the value -- (see ``asString'' in
class object, which gets the 3rd field in the class object, namely
the name of the class.  Note that functions from class Class can't be
invoked in class object because of the forward referencing problems
alluded to earlier).  The second is the loop used in the {\tt typeTest}
function described earlier.

Traditional tasks performed by metaclasses include making new
instances and cloning.  The latter is not currently done in Leda, and
the former is done as a built-in operation, not a metaclass task.

\section{A Type inferencing system}

It is somewhat of a pain to be constantly filling in the type
arguments in parameterized functions or classes.  It would be nice
if the system could automatically infer this information from a function
call, as in ML or similar languages.

%
%	language reference manual for Leda Interpreter
%
\documentstyle[cprog]{book}
\setlength{\textwidth}{5.5in}
\setlength{\oddsidemargin}{0.5in}
\setlength{\evensidemargin}{0.5in}
\setlength{\topmargin}{0.5in}
\pascaltrue
\vbadness 1000
\begin{document}
\title{Leda Language Reference Manual}
\author{Timothy A. Budd \\ Oregon State University \\ Corvallis, Oregon \\ USA}
\maketitle
\chapter{Leda Grammar}\label{gr0}

In this chapter we describe the grammar of the language Leda as it
is currently defined.
Note that the language has evolved over time,
and this grammar differs slightly from the language used in some of
the earlier papers on Leda, especially~\cite{blend} and~\cite{ledadata}.

The grammar will be defined using a combination of BNF and commentary.
Keywords are given in {\bf bold}.  Non-keyword lexical items are
written in {\em italic}.  Nonterminals symbols are written in Roman font.
The symbol $\epsilon$ is used to represent ``nothing'', that is,
an empty sequence of characters.

\section{Overall Program Structure}\label{gr1}

\begin{tabular}{l r l}
program & ::= & declarations body {\bf ;} \\ \\

declarations & ::= & $\epsilon$ \\
& $\mid$ & declarations declaration \\ \\

body & ::= & {\bf begin} statements {\bf end} \\ \\
\end{tabular}

The overall structure of a program is a sequence consisting of zero or more
declarations, followed by a single compound statement.  The compound
statement given in the body is the
statement initially executed when the program is invoked.

Definitions: declaration: Section~\ref{gr2}, statements: Section~\ref{gr6}.

\section{Declarations}\label{gr2}

\begin{tabular}{l r l}
declaration & ::= & {\bf const} constantDefinitions \\
& $\mid$ & {\bf var} variableDefinitions \\
& $\mid$ & {\bf type} typeDefinitions \\
& $\mid$ & functionDeclaration \\
& $\mid$ & classDeclaration \\ \\
constantDefinitions & ::= & constantDefinition \\
& $\mid$ & constantDefinitions constantDefinition \\ \\
constantDefinition & ::= & {\em identifier} {\bf :=} expression {\bf ;} \\ \\
variableDefinitions & ::= & variableDefinition \\
& $\mid$ & variableDefinitions variableDefinition \\ \\
variableDefinition & ::= & identifierList {\bf :} type {\bf ;} \\ \\
identifierList & ::= & {\em identifier} \\
& $\mid$ & identifierList {\bf ,} {\em identifier} \\ \\
typeDefinitions & ::= & typeDefinition \\
& $\mid$ & typeDefinitions typeDefinition \\ \\
typeDefinition & ::= & {\em identifier} {\bf :} type {\bf ;} \\ \\
\end{tabular}

Unlike Pascal, in Leda the different forms of declaration can be listed
in any order, and the same type of declaration section may appear more
than once in a given scope (although individual identifiers cannot
be declared more than once, see below).

The expression assigned to a constant identifier need not be resolvable
at compile time. Such values will be calculated in the order they
are given prior to the execution of any statement in the context
in which they are defined.

All declared identifiers (constant, type, variable, function and class names)
must be unique within the scope in which they declared.

Although type declarations are permitted in Leda they tend to be used
far less frequently than in Pascal, since the vast majority of type
declarations are replaced by classes.

Identifiers declared within a function can only be accessed by statements
defined in the function scope.

Identifiers created within a class definition can be accessed outside
the class scope only if qualified by an expression of the appropriate
class type (see Section~\ref{gr8}).

Definitions:
type: Section~\ref{gr3},
functionDeclaration: Section~\ref{gr4},
classDeclaration: Section~\ref{gr5},
expression: Section~\ref{gr7}.

\section{Types}\label{gr3}

\begin{tabular}{l r l}
type & ::= & {\em identifier} \\
& $\mid$ & {\em identifier} {\bf [} typeList {\bf ]} \\
& $\mid$ & {\bf function} {\bf (} optionalArguments {\bf )} optionalReturnType \\ \\
typeList & ::= & type \\
& $\mid$ & typeList {\bf ,} type \\ \\
optionalArguments & ::= & $\epsilon$ \\
& $\mid$ & formalList \\ \\
formalList & ::= & storageForm type \\
& $\mid$ & formalList {\bf ,} storageForm type \\ \\
storageForm & ::= & $\epsilon$ \\
& $\mid$ & {\bf byRef} \\
& $\mid$ & {\bf byName} \\ \\
optionalReturnType & ::= & $\epsilon$ \\
& $\mid$ & {\bf $-\!{}>$} type \\ \\
\end{tabular}

Types can be divided into class types (which are types generated
automatically by a class definition), resolved types (produced
by binding qualified type parameters on a generic function or class),
or function types.

A type-list used to qualify a generic function or class type
must match in number and compatibility the type arguments given
in the declaration of the underlying type.

The data-type {\tt relation}, used in the back-tracking mechanism, is
the only type predefined by the Leda system (as opposed to being defined in
Leda itself in the standard library).  Although a relation can be
thought of as a form of boolean, in fact a relation is more accurately
described as equivalent to the following definition:
\begin{cprog}

type
	relation : function (relation)->boolean;

\end{cprog}
That is, a relation is a function which takes as argument another relation
and returns a boolean value.   Functions that manipulate relations
are written in Leda and are defined as part of the standard library.
A number of other types are also defined in the standard library.

\section{Function Declarations}\label{gr4}

\begin{tabular}{l r l}
functionDeclaration & ::= & {\bf function} {\em identifier} typeArguments \\
& & valueArguments optionalReturnType {\bf ;} \\
& & declarations body {\bf ;} \\ \\
typeArguments & ::= & $\epsilon$ \\
& $\mid$ & {\bf [} argumentList {\bf ]} \\ \\
valueArguments & ::= & {\bf (} $\;$ {\bf )} \\
& $\mid$ & {\bf (} argumentList {\bf )} \\ \\
argumentList & ::= & storageForm identifierList {\bf :} type \\
& $\mid$ & argumentList {\bf ,} storageForm identifierList {\bf :} type \\ \\
\end{tabular}

A function declaration provides both the function signature (argument
types and return type) and the function body.

The keyword {\bf function} is used to define both functions that
return values and functions that do not return any value (the latter
are sometimes referred to as procedures).

Functions can be made generic by defining type parameters.
If type parameters are omitted then the square brackets are omitted
as well.  However, the parenthesis that surround value parameters
must be written, even if no such parameters are used.

Identifiers declared within a function definition can be accessed only
by statements that originate within the scope of the function.

The meaning of operator symbols can be defined by providing
a function which uses the textual-name for the operator symbol.
See Section~\ref{gr12} for a list of the textual-names for operators.

Function names cannot be overloaded, with the exception of
functions that are defining the meaning of operator symbols, which can
be overloaded only at the global scope.

Definitions:
declarations, body: Section~\ref{gr1},
identifierList: Section~\ref{gr2},
storageForm, type, optionalReturnType: Section~\ref{gr3}.

\section{Class Declarations}\label{gr5}

\begin{tabular}{l r l}
classDeclaration & ::= & classHeading declarations {\bf end} {\bf ;} \\ \\
classHeading & ::= & className {\bf ;} \\
& $\mid$ & className {\bf of} {\em identifier} {\bf ;} \\
& $\mid$ & className {\bf of} {\em identifier} {\bf [} typeList {\bf ]} {\bf ;} \\ \\
className & ::= & {\bf class} {\em identifier} typeArguments \\ \\
\end{tabular}

Class definitions define a new identifier scope, in a similar manner to function
definitions.  In addition, class declarations define the class name
as a new type.

The optional identifier following the keyword {\tt of} must denote
a class name, which is the parent class from which the new class
will inherit.
If the parent class is generic (defined using type-parameters)
then type-parameters must be provided in the bracket-surrounded type argument
list in order to fully resolve the generic class parameters.
Furthermore, such parameters must match in type and in number of
type arguments given in the underlying class declaration.
Such a type-list cannot be used on a non-generic
class.  If no parent class is specified the class {\em object} is assumed.

Definitions:
typeList: Section~\ref{gr3},
typeArguments: Section~\ref{gr4}.

\section{Statements}\label{gr6}

\begin{tabular}{l r l}
statements & ::= & $\epsilon$ \\
& $\mid$ & statements statement {\bf ;} \\ \\
statement & ::= & reference {\bf :=} expression \\
& $\mid$ & {\bf return} \\
& $\mid$ & {\bf return} expression \\
& $\mid$ & {\bf begin} statements {\bf end} \\
& $\mid$ & {\bf if} expression {\bf then} statement \\
& $\mid$ & {\bf if} expression {\bf then} statement {\bf else} statement \\
& $\mid$ & {\bf while} expression {\bf do} statement \\
& $\mid$ & {\bf for} expression {\bf do} statement \\
& $\mid$ & {\bf for} expression {\bf to} expression {\bf do} statement \\
& $\mid$ & {\bf for} reference {\bf :=} expression {\bf to} expression {\bf do} statement \\
& $\mid$ & procedureCall \\
& $\mid$ & $\epsilon$ \\ \\
procedureCall & ::= & functionCall {\bf (} optionalExpressionList {\bf )} \\
& $\mid$ & {\bf cfunction} {\em identifier} {\bf (} optionalExpressionList {\bf )} \\ \\
optionalExpressionList & ::= & $\epsilon$ \\
& $\mid$ & expressionList \\ \\
expressionList & ::= & expression \\
& $\mid$ & expressionList {\bf ,} expression \\ \\
\end{tabular}

The semicolon is used as a statement terminator, not a statement separator.

In an assignment statement, the expression to the right of the assignment
symbol must have a type that is legally assignment-compatible with
the reference described to the left of the assignment symbol.
Assignment is performed using {\em pointer semantics}.
Following the assignment statement the value of the reference on the
left of the assignment arrow is exactly the same as the value on the right
of the assignment arrow.
If the right side is a simple reference, then changes made to one variable
will be reflected in changes made to the other, and vice-versa.
This situation will remain until one or the other reference is subsequently
reassigned, or until one or the other goes out of scope.

A value is assignment-compatible with a variable if
\begin{enumerate}
\item
The type of the value is the same as the type of the variable,
\item
The type of the value is an instance of a subclass for the class associated
with the variable,
\item
The value is the polymorphic constant {\tt NIL},
\item
Both the variable and value are function types, and
\begin{enumerate}
\item
The number of arguments and their storage form match
\item
The type associated with each by-value or by-name parameter in the value is
type-assignment with the equivalent types in the target variable,
\item
The type associated with each by-reference parameter is the same in both
cases (a test which can be implemented by determining that the values are
type-assignable in both directions), and
\item
The type associated with the returned value is type-assignable to the type of
the target.
\end{enumerate}
\end{enumerate}

No automatic conversions are done by Leda,\footnote{There is one small
exception to the rule that Leda performs no automatic conversions.
Relational values will be automatically converted into a boolean value
when they are used in the test expression in an {\tt if} or {\tt while}
statement.} not even the conversion of integer to real.
All such changes must be explicitly specified by the user.
(Mixed mode arithmetic operations are implemented by defining
several forms of the arithmetic operators at the global level).

Parameters which are passed by-name cannot be used as the target of
an assignment, nor can identifiers which are declared as referring to
constant values.
Parameters which are passed by-value are treated
the same as local variables which have been initialized with the value
of the actual argument.  Note that this is a form of assignment,
and thus the comments relative to the semantics of assignment (above)
are true for by-value parameters as well.  This is occasionally a source
of confusion, when an operation that changes the state of a formal parameter
value will be observed to have also changed the state of the actual
parameter value used in a function call.

Return statements are not permitted in the body defined at the global
level.  A return statement with a value can only be used in an
function which returns a value of a type to which the expression
can be assigned.  Conversely, if a return statement is used within
a function that does not produce a value, then no expression
can be used with the return statement.

Return statements are not permitted within the statement associated
with the relational version of the {\tt for} statement.

The expression used to control execution in an {\tt if} statement or
a {\tt while} loop must be either boolean type, or relational type
(which will be converted into boolean type).

An {\tt else} clause is matched with the closest surrounding {\tt if} statement.

The first expression used in the first and second form of the {\tt for}
statement must be of type relation.  In the second form of the {\tt for}
statement the second expression must be type boolean.
In the form of the {\tt for} statement using assignment (the third
form of {\tt for} statement), the reference and all expressions
must be type integer.

Parenthesis must be used in a procedure call, even if no actual
arguments are being passed.

The keyword {\bf cfunction} introduces a run-time system call; an invocation
of an underlying operation that is normally not type-checked.
It is the responsibility of the programmer to ensure that no
type errors can be introduced through the use of this mechanism.
Normally calls on {\bf cfunctions} are hidden within other
functions.  Implementations may make further restrictions on
the use of this mechanism.  (For example, in the Leda Interpreter there
is a fixed set of cfunctions, which can only be modified by recompiling
the interpreter).

In a procedure call the actual arguments being passed must match in number and
compatibility the arguments given in the declaration of the target function.

Definitions:
expression, functionCall : Section~\ref{gr7},
reference: Section~\ref{gr8}.

\section{Expressions}\label{gr7}

\begin{tabular}{l r l}
expression & ::= & andExpression \\
& $\mid$ & expression {\em orSymbol} andExpression \\ \\
andExpression & ::= & notExpression \\
& $\mid$ & andExpression {\em andSymbol} notExpression \\ \\
notExpression & ::= & relationalExpression \\
& $\mid$ & \verb+~+ notExpression \\
& $\mid$ & reference {\bf is} type \\
& $\mid$ & reference {\bf is} type {\bf (} identifierList {\bf )} \\ \\
relationalExpression & ::= & plusExpression \\
& $\mid$ & plusExpression {\em relationalOperator} plusExpression \\ \\
plusExpression & ::= & timesExpression \\
& $\mid$ & plusExpression {\em plusOperator} timesExpression \\ \\
timesExpression & ::= & functionCall \\
& $\mid$ & timesExpression {\em timesOperator} functionCall \\
& $\mid$ & {\em plusOperator} functionCall \\ \\
functionCall & ::= & basicExpression \\
& $\mid$ & functionCall {\bf (} optionalExpressionList {\bf )} \\
& $\mid$ & {\bf cfunction} {\em identifier} {\bf (} optionalExpressionList {\bf )} $->$ type \\ \\
\end{tabular}

The {\em or} symbol is \verb+|+.
The {\em and} symbol is \verb+&+.
The six relational operators are $<$, $<=$, $=$, $<>$, $>=$ and $>$,
The two plus operators are $+$ and $-$.
The three times operators are $*$, $/$ and \verb+%+, the latter denoting
remainder.
The {\bf cfunction} keyword is described in Section~\ref{gr6}.

The {\bf is} keyword introduces a type pattern matching operation.
The expression returns a boolean true value if the expression to the
left of the {\bf is} is an instance or subclass of the type represented
by the identifier to the right of the keyword.  The identifiers in
the optional list following the test must match the data fields
defined for the type.  If the identifier list is present and the test
is successful, the data fields are copied out of the value into the
variables, in effect undoing the actions of the constructor for the class.

In a function call the actual arguments being passed must match in number and
compatibility the arguments given in the declaration of the target function.
Arguments passed to by-reference parameters need not be references; if
non-reference a temporary variable will be generated, the value of
the parameter will be assigned to the temporary, and the reference passed
to the function will be that of the temporary.

Definitions:
type: Section~\ref{gr3},
optionalExpressionList: Section~\ref{gr6},
basicExpression: Section~\ref{gr8}.

\section{Basic Expression}\label{gr8}

\begin{tabular}{l r l}
basicExpression & ::= & reference \\
& $\mid$ & {\em constant} \\
& $\mid$ & {\bf (} expression {\bf )} \\
& $\mid$ & basicExpression {\bf [} typeList {\bf ]} \\
& $\mid$ & {\bf [} expressionList {\bf ]} \\
& $\mid$ & {\bf function} valueArguments optionalReturnType {\bf ;} declarations body \\ \\
reference & ::= & {\em identifier} \\
& $\mid$ & functionCall {\bf .} {\em identifier} \\ \\
\end{tabular}

Constants can be either string constants, integer constants, or
real (floating-point) constants (see Section~\ref{gr11}).

A reference must denote a declared identifier (which can be either
constant, variable, type or function).  In the first
form the identifier must be accessible in the scope containing the reference,
while in the second form the identifier must be accessible in the scope
denoted by the function call, which must be a class type.

A square bracket surrounding a type-list following a basic expression
is used to describe the resolution of qualified types in a function expression.

Square brackets surrounding an expression list is used to define
an array literal.  All the expressions appearing in the list must have
the same type.

Definitions:
declarations, body: Section~\ref{gr1},
typeList: Section~\ref{gr3},
valueArguments, optionalReturnType: Section~\ref{gr4},
expression, expressionList, functionCall: Section~\ref{gr7}.

\section{Comments and Whitespace}\label{gr9}

Any text between matching curly braces is treated as a comment and
will be ignored by the parser.  This is the only legal use for curly
brace characters.

Outside of literal strings, space, tab and newline characters have no
meaning.

\section{Identifiers and Keywords}\label{gr10}

Identifiers must begin with a letter or underscore, and consist of
an arbitrary number of letters, digits, or underscore characters.

In addition to the textual names for operator symbols (see Section~\ref{gr12}),
the following tokens are reserved as keywords, and cannot
be used to define identifiers:

\begin{center}
\begin{tabular}{l l l l l}
begin & byName & byRef & cfunction & class \\
const & defined & do & else & end  \\
for & function & if & include & of \\
return & then & to & type & var \\
while & is \\
\end{tabular}
\end{center}

\section{Textual names for Operator Symbols}\label{gr12}

The meaning of unary and binary operators in any given context is provided
by the implementation of a function with the {\em textual name\/} for the
operator symbol.  The following table gives the various operators
and their associated textual name.

\begin{center}
\begin{tabular}{| l | l |}
\multicolumn{2}{c}{\em binary operators} \\
\hline
symbol & name \\
\hline
\verb-+- & plus \\
\verb+-+ & minus \\
\verb+*+ & times \\
\verb+/+ & divide \\
\verb+%+ & remainder \\
\verb+&+ & and \\
\verb+|+ & or \\
\verb+<+ & less \\
\verb+<=+ & lessEqual \\
\verb+>+ & greater \\
\verb+>=+ & greaterEqual \\
\verb+==+ & sameAs \\
\verb+~=+ & notSameAs \\
\verb+=+ & equals \\
\verb+<>+ & notEquals \\
\hline
\end{tabular} \begin{tabular}{| l | l |}
\multicolumn{2}{c}{\em unary operators} \\
\hline
symbol & name \\
\hline
\verb+~+ & not \\
\verb+-+ & negation \\
\hline
\end{tabular}
\end{center}

Functions that define operator symbols are the only function names that
can be overloaded (be associated with more than one function body defined
in the same scope).  Such overloading can only take place at the
global scope.

\section{Constants}\label{gr11}

There are three types of constants in Leda.  These are string constants,
integer constants, and real (or floating-point) constants.

String constants are surrounded by a matching pair of double-quote
marks.  String constants cannot span multiple input lines, however
the following escape conventions can be used to represent special
characters:

\begin{center}
\begin{tabular}{| l | l |}
\hline
{\em sequence} & {\em meaning} \\
\hline
\verb+\n+ & {\em newline} \\
\verb+\t+ & {\em tab} \\
\verb+\b+ & {\em backspace} \\
\verb+\\+ & {\em backslash} \\
\verb+\"+ & {\em quote mark} \\
\hline
\end{tabular}
\end{center}

There is no separate type for character constants.
Strings of length 1 can be used in place of character values.

Integer constants consist of a sequence of digit characters.
The underlying architecture may impose restrictions on the size
of integer constants that can be recognized.

A real (or floating-point) constant consists of a non-empty sequence
of digit characters, followed by a fractional part and/or an
exponent part.  A fractional part consists of a decimal point
(period) followed by a non-empty sequence of digit characters.
An exponent part consists of the literal character {\bf E} followed
by an optional sign and a non-empty sequence of digit characters.

\chapter{The Standard Library}

In this appendix we describe the functions found in the standard Leda
run-time library.  All of the functionality provided by this library is
written in Leda itself, and is thus available for modification by
the programmer (although care must be exercised when manipulating
some of the more basic classes, such as {\tt object} or {\tt Class}).

\section{The class {\tt object}}\label{std3}

The class {\tt object} is the root class of all Leda entities,
including functions.  It defines two data fields, one of type {\tt Class},
and the second of type {\tt object}.  While used internally, the second field
is currently not used by any user-accessible functions.

Three operations are implemented in class {\tt object}.
The first is the function {\tt asString}, used to convert a value into
a string.  This is used, for example, by the {\tt print} function
(Section~\ref{std1}).  By default objects print simply by printing their
class name.  This function is overridden in a number of classes, such
as integers, string, real, and so on.

The second and third functions define the meaning for the operators
\verb+==+ and $\sim\! =$.  The \verb+==+ operator is implemented so
as to test object identity between the two argument values; that is,
the value returned is true if the two arguments are exactly the same
entity.  This operation is occasionally overridden in subclasses (for
example, it is overridden by integers).  Testing for same identity should
not be confused with testing for equality (next section).  The
$\sim\! =$ operator is defined to be simply the inverse of the \verb+==+
operation.  Thus, $\sim\! =$ will seldom need to be redefined even
if the operation \verb+==+ is overridden.

\begin{cprog}

class object;
var
	classPtr : Class;	{ pointer describing object }
	context : object;	{ pointer to context -- not used }

	function asString ()->string;
	begin		{ return name of our class }
		return cfunction
			Leda_object_at(classPtr, 2)->string;
	end;

	function sameAs (arg : object)->boolean;
	begin		{ true if self and arg are same object }
		return cfunction Leda_object_equals(self, arg)->boolean;
	end;

	function notSameAs (arg : object)->boolean;
	begin		{ inverse of sameAs }
		if self == arg then
			return false;
		return true;
	end;
end;

\end{cprog}

\section{The classes {\tt equality} and {\tt ordered}}

The classes {\tt equality} and {\tt ordered} define protocol for
objects which know how to compare against similar objects for equality,
and comparison using relational operators.   Both classes are parameterized by
a type value which is the type associated with the
right argument in boolean operators.
When subclassing from these classes the class type for the subclass
is normally used as the type argument as well, so that both left
and right arguments in boolean operations have the same type.

The class {\tt equality} simply
provides the meaning for the two operators \verb+=+ and \verb+<>+.
The former is normally redefined in subclasses, although in certain
situations the default meaning, which is identity of the two arguments,
is sufficient.  The latter operation is defined in terms of the former,
so it will seldom need to be redefined in subclasses.

\begin{cprog}

class equality [T : equality];

	function equals (arg : T)->boolean;
	begin
			{  default is simply pointer equality }
		return self == arg;
	end;

	function notEquals (arg : T)->boolean;
	begin
			{ use = which can be redefined }
		if self = arg then
			return false;
		return true;
	end;
end;

\end{cprog}

The class {\tt ordered} provides meaning for the remaining four
relational operators ($<$, $<=$, $>=$, and $>$).
These are defined in terms of each other, so a subclass need only
redefine the meaning of the less-than operator to gain access to the
remaining three.  An additional function {\tt between} is defined which
can be used to determine if a value is between two limits.

\begin{cprog}

class ordered [T : ordered] of equality[T];

	function less (arg : T)->boolean;
	begin
		return false;
	end;

	function lessEqual (arg : T)->boolean;
	begin
		return self < arg | self = arg;
	end;

	function greater (arg : T)->boolean;
	begin
		return ~ (self <= arg);
	end;

	function greaterEqual (arg : T)->boolean;
	begin
		return self > arg | self = arg;
	end;

	function between (low, high : T)->boolean;
	begin
		return low <= self & self <= high;
	end;
end;

\end{cprog}

\section{The class {\tt Class} and the {\tt typeTest} function}

The class {\tt Class} (the initial upper case letter being used to
avoid conflict with the keyword) is used to define protocol for
the object which is the representative of each class.
Classes maintain data fields containing the class name, size,
and parent class.

Class {\tt Class} is a subclass of {\tt ordered}, the relational operators
being used to describe the class/subclass relationship.  That is,
the value of $x < y$ is true if $x$ and $y$ are class values and $x$ is
a (proper) subclass of $y$.  The value of $x <= y$ is true if either $x$ is
the same class as $y$ or $x < y$.

There are three functions implemented in this class.
The first simply overrides the printing function inherited from
class {\tt object}, so that class values respond with their
name.  The second provides implementation for the relational less-than
operator.  Since the default meaning (namely, identity) of the equality 
testing operator is used,
this then implicitly defines all six relational operators.
The third function takes as argument a value, and returns true
if the the value is an instance of the class, or an instance of
a subclass of the class.

\begin{cprog}

class Class of ordered[Class];
var
	name : string;
	size : integer;
	parent : Class;

	function asString ()->string;
	begin
		return name;
	end;

	function less (arg : Class)->boolean;
	begin
		if self == arg then	{ equal, not less }
			return false;
		if parent == arg then
			return true;
		return parent <> self & parent < arg;
	end;

	function isInstance (val : object)->boolean;
	begin
		return val.classPtr <= self;
	end;
end;

\end{cprog}

The latter operation is used in the implementation of the
function {\tt typeTest}, which can be used to not only determine if a value is
or is not an instance of a specific class, but to conver the value into
a form so that it can be assigned to an instance of a class.
(Normally the class in question is a subclass of
the declared type for a variable).  The {\tt typeTest} function
either returns the undefined value {\tt NIL}, which can be assigned to
any variable, or it returns the
first argument value.  The former is used to indicate the value is
not an instance of the class in question, while the latter indicates
it is.  The condition of the result can then be tested using
the predicate {\tt defined}.

\begin{cprog}

function typeTest [T : object] (val : object, aClass : Class)->T;
begin
	if aClass.isInstance(val) then
		return cfunction Leda_object_cast(val)->T;
	return NIL;
end;

\end{cprog}

An example of the use of {\tt typeTest} is found in the function
which overrides the meaning of the \verb+==+ operator in
the class {\tt integer}.  The argument is declared as simply being
an instance of class {\tt object}, whereas the function wishes to
define an alternative meaning if the actual value is an instance
of class {\tt integer}.  See section~\ref{std2}.

\section{The classes {\tt boolean, True and False}}

The class {\tt boolean} defines the meaning of the unary operator \verb+~+
and the binary operators \verb+|+ and \verb+&+.  In the binary operators
the argument is passed by-name, and thus evaluation of the right
argument value will be delayed until it is actually used in the body
of the function.

\begin{cprog}

class boolean of equality[boolean];

	function not ()->boolean;
	begin
		if self then
			return false;
		return true;
	end;

	function or (byName arg : boolean)->boolean;
	begin
			{ if either are true, return true }
		if self then
			return true;
		if arg then
			return true;
		return false;
	end;

	function and (byName arg : boolean)->boolean;
	begin
			{ if both are true, return true }
		if self then
			if arg then
				return true;
		return false;
	end;
end;

\end{cprog}

In actual practice the implementations defined in class {\tt boolean} will
never be executed, as they are all overridden in the pair of classes
{\tt True} and {\tt False}, which are used to create the global constants
{\tt true} and {\tt false}, respectively.  Each class redefines
the conversion to string function (inherited from class {\tt object}),
and the operations defined in class {\tt boolean}.  The effect of
these changes is to avoid the evaluation of the right-hand argument to
binary operators when the result can be determined solely by the left
argument.  Such an evaluation scheme is sometimes referred to as
{\em short circuit evaluation}.

\begin{cprog}

class True of boolean;

	function asString ()->string;
	begin
		return "true";
	end;

	function not ()->boolean;
	begin
		return false;
	end;

	function or (byName arg : boolean)->boolean;
	begin
		return true;
	end;

	function and (byName arg : boolean)->boolean;
	begin
		return arg;
	end;
end;

class False of boolean;

	function asString ()->string;
	begin
		return "false";
	end;

	function not ()->boolean;
	begin
		return true;
	end;

	function or (byName arg : boolean)->boolean;
	begin
		return arg;
	end;

	function and (byName arg : boolean)->boolean;
	begin
		return false;
	end;
end;

\end{cprog}

\section{The class {\tt real}}

The class {\tt real} defines the protocol for real numbers.
Reals are a subclass of class {\tt ordered}, and thus can be used
with relational operators.
The inherited functions {\tt asString}, {\tt equals} and {\tt less}
are all overridden with appropriate definitions.
Reals define the arithmetic operators $+$, $-$, $*$ and $/$, but not
the remainder operator \verb+%+.  The remaining two functions
implement unary negation and the conversion of real values to integer
via truncation.

\begin{cprog}

class real of ordered[real];

	function asString ()->string;
	begin
		return cfunction Leda_real_asString(self)->string;
	end;

	function equals (arg : real)->boolean;
	begin
		return cfunction
			Leda_real_equals(self, arg)->boolean;
	end;

	function less (arg : real)->boolean;
	begin
		return cfunction
			Leda_real_less(self, arg)->boolean;
	end;

	function plus (arg : real)->real;
	begin
		return cfunction
			Leda_real_plus(self, arg)->real;
	end;

	function minus (arg : real)->real;
	begin
		return cfunction
			Leda_real_minus(self, arg)->real;
	end;

	function times (arg : real)->real;
	begin
		return cfunction
			Leda_real_times(self, arg)->real;
	end;

	function divide (arg : real)->real;
	begin
		return cfunction
			Leda_real_divide(self, arg)->real;
	end;

	function negation ()->real;
	begin
		return 0.0 - self;
	end;

	function asInteger()->integer;
	begin
		return cfunction
			Leda_real_asInteger(self)->integer;
	end;
end;

\end{cprog}

There are (almost) no automatic coercions or conversions performed by
Leda.\footnote{The one exception being the conversion of relation values into
booleans in the expression portion of an {\tt if} or {\tt while} statement.}
Mixed mode arithmetic is implemented by overloaded versions of
the arithmetic operators, defined in the global scope:

\begin{cprog}

function plus (left : integer, right : real)->real;
begin
	return left.asReal() + right;
end;

function plus (left : real, right : integer)->real;
begin
	return left + right.asReal();
end;

\end{cprog}

\section{The class {\tt integer}}\label{std2}

The class {\tt integer} defines protocol for the manipulation of
integer objects.  The underlying computer architecture may place
limitations on the size of integers that can be held in an instance
of this class.  (The creation of infinite precision integers is, however,
an interesting and useful programming exercise.  Since the standard
library is written in Leda itself such values can even be easily
integrated into the rest of the system).

The class {\tt integer} inherits from class {\tt ordered}, and redefines
the relational operators, as well as the conversion to string
function inherited from class {\tt object}.  The identity testing
operator \verb+==+ is redefined in class {\tt integer} so that two
integer values with the same magnitude will test equal.
Doing so requires testing the right argument, which is declared simply
as an instance of class {\tt object}.

The binary operators \verb+|+ and \verb+&+ are defined so as to mean
bit-wise boolean operations, as in the unary operator \verb+~+, which
produces a result by inverting all bits in the argument.

The arithmetic operators $+$, $-$, $*$, $/$ and \verb+%+ are given
their expected meaning, the latter being defined as the remainder which
remains after the left argument is divided by the right.

The final two operations implemented in the class perform unary negation
and the conversion of integer values into real numbers.

\begin{cprog}

class integer of ordered[integer];

	function asString ()->string;
	begin
		return cfunction Leda_integer_asString(self)->string;
	end;

	function sameAs (arg : object)->boolean;
	var
		argInt : integer;
	begin		
			{ if arg is an integer, use int compare }
		argInt := typeTest[integer](arg, integer);
		if defined(argInt) then
			return self = argInt;
			{ not equal to anything but integer }
		return false;
	end;

	function equals (arg : integer)->boolean;
	begin
		return cfunction
			Leda_integer_equals(self, arg)->boolean;
	end;

	function less (arg : integer)->boolean;
	begin
		return cfunction
			Leda_integer_less(self, arg)->boolean;
	end;

	function or (arg : integer)->integer;
	begin
		return cfunction
			Leda_integer_or(self, arg)->integer;
	end;

	function and (arg : integer)->integer;
	begin
		return cfunction
			Leda_integer_and(self, arg)->integer;
	end;

	function not ()->integer;
	begin
		return cfunction Leda_integer_not(self)->integer;
	end;

	function plus (arg : integer)->integer;
	begin
		return cfunction
			Leda_integer_plus(self, arg)->integer;
	end;

	function minus (arg : integer)->integer;
	begin
		return cfunction
			Leda_integer_minus(self, arg)->integer;
	end;

	function times (arg : integer)->integer;
	begin
		return cfunction
			Leda_integer_times(self, arg)->integer;
	end;

	function divide (arg : integer)->integer;
	begin
		return cfunction
			Leda_integer_divide(self, arg)->integer;
	end;

	function remainder (arg : integer)->integer;
	var
		result : integer;
	begin
		result := self - (self / arg) * arg;
		if result < 0 then
			result := result + arg;
		return result;
	end;

	function asReal ()->real;
	begin
		return cfunction
			Leda_integer_asReal(self)->real;
	end;

	function negation ()->integer;
	begin
		return 0 - self;
	end;
end;

\end{cprog}

\section{The class {\tt string}}

The class {\tt string} implements operations used in the manipulation
of text values.  Strings are a subclass of ordered, and define
the ordering operations as lexicographic (dictionary-style)
sequencing.  The plus operation is used to concatenate
two string values together -- the second argument is converted
into a string using the function {\tt asString} if it is not
already a string.  The function {\tt length} is used to determine
the number of characters in a string.  The function {\tt subString}
is used to extract a subportion of a string; the two arguments represent
the starting location and length of the subtext.  If the first
argument is negative it is interpreted as an index from the right side.
The function {\tt index} takes a second string as argument and
returns the position in the receiver at which the argument string
first appears, returning {\tt NIL} if the argument does not appear
in the string at all.

\begin{cprog}

class string of ordered[string];
	
	function asString ()->string;
	begin
		return self;
	end;

	function equals (arg : string)->boolean;
	begin
		return 0 = cfunction
			Leda_string_compare(self, arg)->integer;
	end;

	function less (arg : string)->boolean;
	begin
		return 0 > cfunction
			Leda_string_compare(self, arg)->integer;
	end;

	function plus (arg : object)->string;
	begin
		return cfunction
			Leda_string_concat(self, arg.asString())->string;
	end;

	function print ();
	begin
		cfunction Leda_string_print(self);
	end;

	function length ()->integer;
	begin
		return cfunction Leda_string_length(self)->integer;
	end;

	function subString (start, len : integer)->string;
	const
		selfLength := length();
	begin
		start := start % selfLength;
		if len < 0 then
			len := 0;
		if start + len > selfLength then
			len := selfLength - start;
		return cfunction
			Leda_string_substring(self, start, len)->string;
	end;

	function index (pattern : string)->integer;
	const
		patternLength := pattern.length();
	var
		position : integer;
	begin
		for position := 0 to length() do
			if pattern = subString(position, patternLength) then
				return position;
		return NIL;
	end;

end;

\end{cprog}

\section{The class {\tt array}}

The {\tt array} data structure is the only collection data type defined
in the standard library.  An array is a fixed-length indexed
collection of elements having the same type.  The function {\tt size}
returns the number of elements held in the array.  Elements are
placed into an array using the function {\tt atPut}, and extracted
from the array using the function {\tt at}.  The relation {\tt items}
is used to iterate over the values held in an array.

New arrays can be generated using the function {\tt newArray}.
The user must supply both the lower and upper index values for the array.
A literal array is generated by surrounding a list of similarly-typed
values with square brackets.  In literal arrays index values
always begin at zero.

\begin{cprog}

class array [T : object] of equality[array];
var
	lowerBound : integer;
	higherBound : integer;
	data : object;

	function size ()->integer;
	begin		{ return number of elements in collection }
		return 1 + (higherBound - lowerBound);
	end;

	function at (index : integer)->T;
	begin
		if index.between(lowerBound, higherBound) then
			return cfunction
				Leda_object_at(data, index - lowerBound)->T
		else
			return NIL;
	end;

	function atPut (index : integer, newVal : T);
	begin
		if index.between(lowerBound, higherBound) then
			cfunction Leda_object_atPut(data, index - lowerBound, newVal);
	end;

	function items (byRef val : T)->relation;
	var
		index : integer;
	begin
		return (lowerBound <= higherBound) &
			integerRange(lowerBound, higherBound, 1, index) &
			val <- at(index);
	end;

	function asString ()->string;
	var
		result : string;
		v : T;
	begin
		result := "[";
		for items(v) do
			result := result + v + " ";
		return result + "]";
	end;
end;

function newArray [T : object] (lb, hb : integer)->array[T];
begin
	if (hb > lb) then
		return array[T](lb, hb,
			cfunction Leda_object_allocate((hb-lb)+1)->object)
	else
		return NIL;
end;

\end{cprog}

\section{Testing for Definition}

The function {\tt defined} can be used to test whether a value is
defined.  It does this by simply comparing against the value {\tt NIL}.

\begin{cprog}

function defined (arg : object)->boolean;
begin
	return ~ cfunction Leda_object_equals(NIL, arg)->boolean;
end;

\end{cprog}

For efficiency reasons some implementations may elect to implement
this function as a built-in operation rather than as Leda code in the
standard library.
The only noticable difference (other than efficiency) would be that
{\tt defined} then becomes a keyword, and can not be used in any other
context.

\section{Input and Output}\label{std1}

Output to the standard output device is performed using
the function {\tt print}.  The argument to this function is declared
as object, and thus can be any value.  It is converted into a string
using the function {\tt asString} (see Section~\ref{std3}), and
the resulting string is then displayed.  If the argument is undefined
a literal value is displayed.

\begin{cprog}

function print (arg : object);
begin
	if defined(arg) then
			{ convert to string, then print }
		arg.asString().print()
	else
		"(undefined)".print();
end;

\end{cprog}

Input is performed line by line.  The function {\tt readLine} reads
a single line from the standard input, assigning it to a by-reference
parameter.  A boolean true value is returned if input is successful,
while a boolean false value is returned on end of input.

\begin{cprog}

function readLine (byRef line : string)->boolean;
begin
	line := cfunction Leda_stdin_read()->string;
	return defined(line);
end;

\end{cprog}

\section{Relations}

From the programmers point of view, relations are defined using
the relational assignment operator (\verb+<-+), conjunctions and
disjunctions, and boolean functions.  The relational assignment
operator is implemented using the following function:

\begin{cprog}

function Leda_arrow (byRef left : object, right : object)->relation;
begin
	return function(future : relation)->boolean;
		var
			save : object;
	begin
			{ save the current value of the left argument }
		save := left;
			{ then change it to the right }
		left := right;
			{ try the task to be done }
		if future(trueRelation) then
				{ worked, report success }
			return true;
			{ did not work, restore the old value }
		left := save;
		return false;
	end;
end;

\end{cprog}

Conjunction of two relations is implemented by the following:

\begin{cprog}

function and (left : relation, byName right : relation)->relation;
begin
		{ conjunction of two relations -- do one after the other }
	return function(future : relation)->boolean;
		begin
			return left(function(f : relation)->boolean;
				begin
					return right(future);
				end);
		end;
end;

\end{cprog}

The backtracking mechanism is found in the implementation of
the disjunction operator:

\begin{cprog}

function or (left : relation, byName right : relation)->relation;
begin
		{ disjunction of two relations -- do backtracking }
	return function(future : relation)->boolean;
		begin
			if left(future) then return true;
			return right(future);
		end;
end;

\end{cprog}

Conversions between relations and booleans are performed using
a pair of functions:

\begin{cprog}

function booleanAsRelation (byName x : boolean)->relation;
begin
		{ convert boolean into a relation }
	return function(future : relation)->boolean;
		begin
			return x & future(trueRelation);
		end;
end;

function relationAsBoolean (future : relation)->boolean;
begin
		{ convert relation into a boolean }
	return future(trueRelation);
end;

\end{cprog}

The function {\tt unify} is used as a simple approximation to
Prolog-style unification (more complex unification techniques
can also be written in Leda, but are generally not necessary).
It attempts to make the left argument the same as the right,
either because they are both defined and equal, or by assigning
one the value of the other.

\begin{cprog}

function unify [T : equality] (byRef left : T, right : T)->relation;
begin
	if defined(left) then
		return left = right
	else
		return left <- right;
end;

\end{cprog}

Finally, the function {\tt integerRange} defines a simple
relation which will generate values from a range of integers.
The starting and ending values and the step amount must be specified.
The ending value must be reached exactly, or an infinite regression will
result.

\begin{cprog}

function integerRange
	(low, high, step : integer, byRef ident : integer)->relation;
begin
	return ident <- low
		| (low <> high) & integerRange(low + step, high, step, ident);
end;

\end{cprog}

\chapter{Unresolved Language Issues in Leda}\label{w0}

Although the Leda language as described in this manual is now reasonably
mature, there remain a few issues that are frequently
sources of error or frustration, and are likely targets for future changes
in the language (or in the next generation multiparadigm language).
In this chapter we will describe the most critical of these issues.
(In addition, the current Leda implementation makes a number of limiting
assumptions, thus supporting only a subset of the language defined
in Chapter~\ref{gr0}.
This, however, is simply an issue of implementation, and not language
design.)

In reading the following, keep in mind that an overriding design
objective has been to keep the language as simple as possible, but no
simpler.  Many issues could be addressed by introducing new {\em mechanisms},
and one must always weight the cost (both conceptual and practical) against
the benefits.

\section{Semantics of Assignment and Parameter Passing}

In several places we have
described the particular semantics of assignment and parameter passing
in Leda.  There seems little chance that the semantics of assignment
will be changed, but there is some possibility that the technique used
in parameter passing can be modified, particularly if a mechanism
for cloning values is introduced (see Section~\ref{w1}).
The principle problem with current parameter passing mechanism is that
it does not prevent pass-by-value parameters from being modified from within
a called procedure.  There have been various proposals made to address this
difficulty:

\begin{itemize}
\item
Make a clone or copy of a value before passing as a by-value parameter.
(Requires a general cloning facility).
\item
Make by-value parameters constant, and forbid operations on constant
values.  (Currently, operations on constant values are not forbidden,
only assignment to constants).  This seems too restrictive, as many
existing programs would then cease to work.
\item
Somehow characterize which functions are constant preserving and
which are not, and only allow constant preserving functions to
be performed on by-value parameters.
It is not at all clear, however, how to discover which functions
preserve constant values, and which do not.
\end{itemize}

\section{Modularization Facilities}

Currently a Leda program is simply one large file.  There are ``include''
facilities, of course, so that data structures can be moved from one
application to another, but no mechanism for separate compilation.

\section{Restricted Scoping}

Currently all variables declared within a class scope are accessible
using the dot operator.  Good style dictates that such values
should never be modified outside the class functions, but the
language does not enforce this restriction.  A proposed change
would be to include C++ style modifiers for declarations which would
separate those values which were purely local to a class from those
which are permitted to be modified outside the class.

\section{Shared and Overridden Data Fields}

In an earlier version of Leda it was possible to declare data fields which
were shared by all instances of a class.  It was also possible to
override data fields declared in parent classes, in much the same
way that methods are overridden.  Both features were little used,
and due to difficulties in syntax or implementation, both have been
dropped from the current version of the language.

\section{The Subscript operator}

It is conventional in programming languages, going back to the days of
Fortran and ALGOL-60, to use braces of some sort to indicate
array subscripts.  Should Leda follow this tradition?

\begin{itemize}
\item
Arrays are currently only weakly part of the ``language'', if by this
one means the syntax recognized by the parser.  Instead, arrays are
simply a data structure that happens to be implemented in the standard
library.  (This is not entirely true, since array literals require
a special syntax, and thus the parser does need to know about the
``Array'' data type in order to give a type to these literals).
\item
Is it right to complicate the syntax of the language (by adding array
expressions) for just a single data structure?  Of course, one could
say that this is not really just for arrays -- onE could add productions
like:
\begin{cprog}

basicExpression:
	basicExpression LEFTBRACKET expression-list RIGHTBRACKET

\end{cprog}\noindent
and say this is simply a shorthand for the function call
\begin{cprog}

	expression.subscript(arguments);

\end{cprog}
where the number of expressions must match the number of arguments provided
for whatever the subscript operator is found in the class of the expression.
(This would permit users to define their own array-like things, such
as matrices or dictionaries).
\item
Presuming one did do something like the previous, what should the return
type be for the subscript operator?  Note that the semantics of
subscripts are different depending upon whether they appear as the
target of an assignment or as an expression.
There seems to be two choices -- either have separate operators
for these two operations (similar to the ``at'' and ``atPut'' functions
now in use in Leda, admittedly copied from Smalltalk), or introduce
some mechanism for explicitly
dealing with references as values (as in C++).  Neither of these seems
particularly appealing.
\end{itemize}

\section{Overloaded operators and/or functions}

Currently there is only a minimal amount of overloading of function
names permitted in Leda.  Multiple copies of operator
definitions (plus, times, ``and'' and ``or'' and the like) are permitted
at the global scope, and nowhere else.  Types of overloading that
are {\em not} permitted include:
\begin{itemize}
\item
overloading of other function names at the global scope
\item
overloading of operators in local scopes (either within classes or within
functions)
\item
overloading of other function names in local scopes
\end{itemize}

The argument for outlawing each of these is the same.  That is, that
currently if {\tt bar} is a function in scope {\tt foo}, the expression
{\tt foo.bar} has a single well defined and unambiguous meaning.
This is a value of type function (technically, a closure), and can
be assigned to a variable, passed as an argument, and so on.  In fact,
all procedure calls, such as:
\begin{cprog}

	foo.bar(x, y, z)

\end{cprog}\noindent
are actually treated in two steps.  First, the expression foo.bar is
evaluated, resulting in a closure value.  Then this closure is
invoked as a function.
Introducing any of the overloadings that are currently not permitted
would destroy this property.

(It is sometimes argued that at least operators should be permitted to
be overloaded in class scopes, since operators are never invoked without
arguments, but this is not true; several places in the book I refer
to expression such as ``integer.plus'' and expect it to mean the
binary function named plus defined in the class integer.  Allowing
overloading would make such an expression ambiguous.)

\section{Nested Classes}

This is probably less an open point, and more simply a bug in the
current language implementation.  A glance at the current Leda
grammar will indicate that I had originally intended for classes
and functions to be permitted to be nested arbitrarily -- functions
within functions, functions within classes, classes within functions,
or classes within classes.  While pondering the question of forward
references (see next section) the following awkward example was discovered: 

\begin{cprog}

class A;

	function foo();
	begin
		...
	end;

	class B of A;
	 ...
	end;

	function bar();
	begin
		...
	end;
end;

\end{cprog}

That is, within class A a new class B is defined, which subclasses from A.
Now, currently the Leda language has been carefully crafted so that it
can be parsed and compiled in one pass.  This being the case, do
instances of B inherit the function foo from class A?  what about
the function bar?  If an instance of B is assigned to a variable
declared as maintaining an A, can we perform the bar function on
the result?  There are a number of technical issues that complicate
this (meaning, subclasses are basically an extension of the parent
class, and one can't extend something that hasn't been completely
constructed yet).  Probably the most logical thing is to rule out
completely the type of construct shown here.

\section{Forward References for Classes or Functions}

In order versions of the Leda language I separated class definition
from method definition.  In the latest revision I have used the
BETA idea (in fact this actually goes all the way back to SIMULA) of just
putting everything together into one big structure.
While this greatly simplified the language description, it did have
the disadvantage that it makes it
difficult to write mutually recursive classes.

Consider first the analogy to mutually recursive functions.
By mutually recursive functions I mean that a function
A can call function B, which is some cases can call function A again.
Mutually recursive functions can sometimes (but not always) be handed
by nesting, as in
\begin{cprog}

	function A();

		function B();
		begin
			...
			A();
		end;

	begin
		...
		B();
		...
	end;	

\end{cprog}
It was the consideration of the analogy to classes that led me to discover
the problem with nested classes described in the previous section.

Another common way to get around the problem of mutually recursive
functions, when nesting doesn't work, is to create a variable of
type function, which is later defined:

\begin{cprog}

	var
		A : function();

	function B();
	begin
		...
		A();
		...
	end;

	function C();
	begin
		...
		B();
		...
	end;

	begin
		A := C;
	end;

\end{cprog}
This is a somewhat primitive facility that nevertheless permits functions
to be forward referenced (that is, named before they are defined).
A more general forward reference mechanism for functions could probably
be defined and easily implemented, but I don't know if the payoff is
worth the effort, because the situation is not that common,
either nesting or the variable trick is usually sufficient, and in
any case neither gives us any hint as to how to handle forward
referencing of mutually recursive classes.

The equivalent to the mutually recursive functions in classes is
a class A which in one of its methods creates and manipulates
an instance of class B,
and a class B in which in one of its methods creates and manipulates
an instance of class A.  In the old Leda (and in C++, Object Pascal, and
other languages which separate the class definition from the method
definitions) the solution is clear.  Give the class definition for A,
followed by the class definition for B, followed by the methods for A,
followed by the methods for B.  By the time the methods are seen
both class definitions have been read, so both are known.

\section{Copies or Clones}\label{w1}

A common operation is to make a copy, or clone, of a value.
Of course, it is easy to define a clone method in any {\em particular} class
to do this operation.  What is harder is to try to develop a system-wide
function or method to generalize this.

The most obvious approach would be to have a function defined in class
object, which would thus be inherited by all values.  Classes which
wanted to change the meaning of this (such as integers or arrays)
could then easily override the method.  But the difficulty with this
is the question -- what should be the return type associated with this
function?

A similar situation occurs with regards to the problem of ``reverse
polymorphism''.  That is, taking a value known to be of some superclass,
and trying to discover whether in fact it is an element of a subclass.
In the current Leda this is accomplished in a somewhat clumsy fashion,
through a parameterized function named {\tt typeTest} which is defined
as follows:

\begin{cprog}

function typeTest [T : object] (val : object, aClass : Class)->T;
begin
	if aClass.isInstance(val) then
		return cfunction Leda_object_cast(val)->T;
	return NIL;
end;

\end{cprog}

What is clumsy about this is that the class name must be both passed
as a type-parameter (which is used by the parser, not during execution)
and as a value parameter (used by the execution of the procedure).
But this seems to be the cost for having the mechanism defined
entirely in Leda, and not handed as a special case by the parser.

But this also means that subclasses can not define their own version of
``typeTest'', as would seemingly be necessary to make an equivalent copy
procedure.

\section{Enumerated Datatypes}

Older versions of the Leda language had enumerated datatypes -- differing
from their PASCAL or C cousins only in that they printed using their
symbolic form and could be iterated over using a relation.
But the amount of work needed by the parser and the changes to the
grammar and the run-time system necessitated by the introduction of
these features seems all out of proportion to their utility.

It also seemed to be the case that enumerated datatypes in previous
versions of the language were primitive, and not easily extensible (by
adding new functionality, for example).

In the latest Leda I have found a literal array of strings to be
almost as easy to use, and does not require any new language features.

(A half-way measure might be to introduce symbols, as in Lisp or Smalltalk.
The equivalent to an enumerated datatype would then be a literal array
of symbols, rather than strings).

\section{User Defined Constructors}

In my previous extensive work in Smalltalk and C++ I found that the large
majority of time (perhaps as high as 75\%) that constructors
were used, they simply initialized the data fields in the newly
generated object.  This is why in Leda this is the default behavior,
and doesn't require any additional effort on the part of the programmer.
Nevertheless, in the remaining 25\% of the time it would be nice
to have more general constructor facilities that could perform
actions other than simply initialization.

I have toyed with a couple of idea.  The BETA-like solution would be
to make classes more like functions, as in
\begin{cprog}

	class A(a, b, c : x) of B(a, b);
		...
	begin
		{ do initialization here }
	end;

\end{cprog}

This makes the class heading somewhat complicated to look at, (and even
harder to parse -- for example the parent field is now an expression,
not a type) but is a nice general mechanism.
Unfortunately, it limits one to a single constructor for every class.

Another possible solution is to permit some sort of special ``functions''
to appear within a class, such as is done in C++, where a function
with the same name as the class is treated differently from other
functions.  This allows multiple constructors for a single class,
but seems in other respects somewhat less elegant.

\section{Metaclasses}

Are classes objects or not?  If the answer is Yes, then what is their type?
This is commonly called metaclass programming, and is somewhat popular at
present.  I've always had a slight suspicion of metaclass programming,
thinking that the mechanism is far too complex for the benefits involved.
Leda does have class values -- see the description of class object
and class Class in the standard library.
Every object has a field of type Class,
which is holding the name of the class, its size, the pointer to its
parent class, and methods.  (The
latter don't actually appear in the class description).

Currently only two things are done with this information.
The first is the display
of the class name when printing the value -- (see ``asString'' in
class object, which gets the 3rd field in the class object, namely
the name of the class.  Note that functions from class Class can't be
invoked in class object because of the forward referencing problems
alluded to earlier).  The second is the loop used in the {\tt typeTest}
function described earlier.

Traditional tasks performed by metaclasses include making new
instances and cloning.  The latter is not currently done in Leda, and
the former is done as a built-in operation, not a metaclass task.

\section{A Type inferencing system}

It is somewhat of a pain to be constantly filling in the type
arguments in parameterized functions or classes.  It would be nice
if the system could automatically infer this information from a function
call, as in ML or similar languages.

\chapter{Restrictions in the Current Implementation}

The current implementation is simplified by restricting the language in
a number of ways:
\begin{itemize}
\item
Classes can only be defined at the global level, not within functions.
\item
Classes cannot be nested.
\item
Constants can only be defined in functions, not in classes.
\end{itemize}

\begin{thebibliography}{Knuth73}
\bibitem[Budd 91a]{blend}
Timothy A. Budd,
``Blending Imperative and Relational Programming,''
{\em IEEE Software}, 8(1):58--65, January 1991.

\bibitem[Budd 91b]{platypus}
Timothy A. Budd, {\em An Introduction to Object-Oriented Programming},
Addison-Wesley, Reading, Mass., 1991.

\bibitem[Budd 92]{ledadata}
Timothy A. Budd,
``Multiparadigm Data Structures in Leda,''
{\em Proceedings of the 1992 International Conference on Computer Languages},
IEEE Computer Society Press, Oakland, California, pp 165--173, April 1992.
\end{thebibliography}
\end{document}






\chapter{Leda Grammar}\label{gr0}

In this chapter we describe the grammar of the language Leda as it
is currently defined.
Note that the language has evolved over time,
and this grammar differs slightly from the language used in some of
the earlier papers on Leda, especially~\cite{blend} and~\cite{ledadata}.

The grammar will be defined using a combination of BNF and commentary.
Keywords are given in {\bf bold}.  Non-keyword lexical items are
written in {\em italic}.  Nonterminals symbols are written in Roman font.
The symbol $\epsilon$ is used to represent ``nothing'', that is,
an empty sequence of characters.

\section{Overall Program Structure}\label{gr1}

\begin{tabular}{l r l}
program & ::= & declarations body {\bf ;} \\ \\

declarations & ::= & $\epsilon$ \\
& $\mid$ & declarations declaration \\ \\

body & ::= & {\bf begin} statements {\bf end} \\ \\
\end{tabular}

The overall structure of a program is a sequence consisting of zero or more
declarations, followed by a single compound statement.  The compound
statement given in the body is the
statement initially executed when the program is invoked.

Definitions: declaration: Section~\ref{gr2}, statements: Section~\ref{gr6}.

\section{Declarations}\label{gr2}

\begin{tabular}{l r l}
declaration & ::= & {\bf const} constantDefinitions \\
& $\mid$ & {\bf var} variableDefinitions \\
& $\mid$ & {\bf type} typeDefinitions \\
& $\mid$ & functionDeclaration \\
& $\mid$ & classDeclaration \\ \\
constantDefinitions & ::= & constantDefinition \\
& $\mid$ & constantDefinitions constantDefinition \\ \\
constantDefinition & ::= & {\em identifier} {\bf :=} expression {\bf ;} \\ \\
variableDefinitions & ::= & variableDefinition \\
& $\mid$ & variableDefinitions variableDefinition \\ \\
variableDefinition & ::= & identifierList {\bf :} type {\bf ;} \\ \\
identifierList & ::= & {\em identifier} \\
& $\mid$ & identifierList {\bf ,} {\em identifier} \\ \\
typeDefinitions & ::= & typeDefinition \\
& $\mid$ & typeDefinitions typeDefinition \\ \\
typeDefinition & ::= & {\em identifier} {\bf :} type {\bf ;} \\ \\
\end{tabular}

Unlike Pascal, in Leda the different forms of declaration can be listed
in any order, and the same type of declaration section may appear more
than once in a given scope (although individual identifiers cannot
be declared more than once, see below).

The expression assigned to a constant identifier need not be resolvable
at compile time. Such values will be calculated in the order they
are given prior to the execution of any statement in the context
in which they are defined.

All declared identifiers (constant, type, variable, function and class names)
must be unique within the scope in which they declared.

Although type declarations are permitted in Leda they tend to be used
far less frequently than in Pascal, since the vast majority of type
declarations are replaced by classes.

Identifiers declared within a function can only be accessed by statements
defined in the function scope.

Identifiers created within a class definition can be accessed outside
the class scope only if qualified by an expression of the appropriate
class type (see Section~\ref{gr8}).

Definitions:
type: Section~\ref{gr3},
functionDeclaration: Section~\ref{gr4},
classDeclaration: Section~\ref{gr5},
expression: Section~\ref{gr7}.

\section{Types}\label{gr3}

\begin{tabular}{l r l}
type & ::= & {\em identifier} \\
& $\mid$ & {\em identifier} {\bf [} typeList {\bf ]} \\
& $\mid$ & {\bf function} {\bf (} optionalArguments {\bf )} optionalReturnType \\ \\
typeList & ::= & type \\
& $\mid$ & typeList {\bf ,} type \\ \\
optionalArguments & ::= & $\epsilon$ \\
& $\mid$ & formalList \\ \\
formalList & ::= & storageForm type \\
& $\mid$ & formalList {\bf ,} storageForm type \\ \\
storageForm & ::= & $\epsilon$ \\
& $\mid$ & {\bf byRef} \\
& $\mid$ & {\bf byName} \\ \\
optionalReturnType & ::= & $\epsilon$ \\
& $\mid$ & {\bf $-\!{}>$} type \\ \\
\end{tabular}

Types can be divided into class types (which are types generated
automatically by a class definition), resolved types (produced
by binding qualified type parameters on a generic function or class),
or function types.

A type-list used to qualify a generic function or class type
must match in number and compatibility the type arguments given
in the declaration of the underlying type.

The data-type {\tt relation}, used in the back-tracking mechanism, is
the only type predefined by the Leda system (as opposed to being defined in
Leda itself in the standard library).  Although a relation can be
thought of as a form of boolean, in fact a relation is more accurately
described as equivalent to the following definition:
\begin{cprog}

type
	relation : function (relation)->boolean;

\end{cprog}
That is, a relation is a function which takes as argument another relation
and returns a boolean value.   Functions that manipulate relations
are written in Leda and are defined as part of the standard library.
A number of other types are also defined in the standard library.

\section{Function Declarations}\label{gr4}

\begin{tabular}{l r l}
functionDeclaration & ::= & {\bf function} {\em identifier} typeArguments \\
& & valueArguments optionalReturnType {\bf ;} \\
& & declarations body {\bf ;} \\ \\
typeArguments & ::= & $\epsilon$ \\
& $\mid$ & {\bf [} argumentList {\bf ]} \\ \\
valueArguments & ::= & {\bf (} $\;$ {\bf )} \\
& $\mid$ & {\bf (} argumentList {\bf )} \\ \\
argumentList & ::= & storageForm identifierList {\bf :} type \\
& $\mid$ & argumentList {\bf ,} storageForm identifierList {\bf :} type \\ \\
\end{tabular}

A function declaration provides both the function signature (argument
types and return type) and the function body.

The keyword {\bf function} is used to define both functions that
return values and functions that do not return any value (the latter
are sometimes referred to as procedures).

Functions can be made generic by defining type parameters.
If type parameters are omitted then the square brackets are omitted
as well.  However, the parenthesis that surround value parameters
must be written, even if no such parameters are used.

Identifiers declared within a function definition can be accessed only
by statements that originate within the scope of the function.

The meaning of operator symbols can be defined by providing
a function which uses the textual-name for the operator symbol.
See Section~\ref{gr12} for a list of the textual-names for operators.

Function names cannot be overloaded, with the exception of
functions that are defining the meaning of operator symbols, which can
be overloaded only at the global scope.

Definitions:
declarations, body: Section~\ref{gr1},
identifierList: Section~\ref{gr2},
storageForm, type, optionalReturnType: Section~\ref{gr3}.

\section{Class Declarations}\label{gr5}

\begin{tabular}{l r l}
classDeclaration & ::= & classHeading declarations {\bf end} {\bf ;} \\ \\
classHeading & ::= & className {\bf ;} \\
& $\mid$ & className {\bf of} {\em identifier} {\bf ;} \\
& $\mid$ & className {\bf of} {\em identifier} {\bf [} typeList {\bf ]} {\bf ;} \\ \\
className & ::= & {\bf class} {\em identifier} typeArguments \\ \\
\end{tabular}

Class definitions define a new identifier scope, in a similar manner to function
definitions.  In addition, class declarations define the class name
as a new type.

The optional identifier following the keyword {\tt of} must denote
a class name, which is the parent class from which the new class
will inherit.
If the parent class is generic (defined using type-parameters)
then type-parameters must be provided in the bracket-surrounded type argument
list in order to fully resolve the generic class parameters.
Furthermore, such parameters must match in type and in number of
type arguments given in the underlying class declaration.
Such a type-list cannot be used on a non-generic
class.  If no parent class is specified the class {\em object} is assumed.

Definitions:
typeList: Section~\ref{gr3},
typeArguments: Section~\ref{gr4}.

\section{Statements}\label{gr6}

\begin{tabular}{l r l}
statements & ::= & $\epsilon$ \\
& $\mid$ & statements statement {\bf ;} \\ \\
statement & ::= & reference {\bf :=} expression \\
& $\mid$ & {\bf return} \\
& $\mid$ & {\bf return} expression \\
& $\mid$ & {\bf begin} statements {\bf end} \\
& $\mid$ & {\bf if} expression {\bf then} statement \\
& $\mid$ & {\bf if} expression {\bf then} statement {\bf else} statement \\
& $\mid$ & {\bf while} expression {\bf do} statement \\
& $\mid$ & {\bf for} expression {\bf do} statement \\
& $\mid$ & {\bf for} expression {\bf to} expression {\bf do} statement \\
& $\mid$ & {\bf for} reference {\bf :=} expression {\bf to} expression {\bf do} statement \\
& $\mid$ & procedureCall \\
& $\mid$ & $\epsilon$ \\ \\
procedureCall & ::= & functionCall {\bf (} optionalExpressionList {\bf )} \\
& $\mid$ & {\bf cfunction} {\em identifier} {\bf (} optionalExpressionList {\bf )} \\ \\
optionalExpressionList & ::= & $\epsilon$ \\
& $\mid$ & expressionList \\ \\
expressionList & ::= & expression \\
& $\mid$ & expressionList {\bf ,} expression \\ \\
\end{tabular}

The semicolon is used as a statement terminator, not a statement separator.

In an assignment statement, the expression to the right of the assignment
symbol must have a type that is legally assignment-compatible with
the reference described to the left of the assignment symbol.
Assignment is performed using {\em pointer semantics}.
Following the assignment statement the value of the reference on the
left of the assignment arrow is exactly the same as the value on the right
of the assignment arrow.
If the right side is a simple reference, then changes made to one variable
will be reflected in changes made to the other, and vice-versa.
This situation will remain until one or the other reference is subsequently
reassigned, or until one or the other goes out of scope.

A value is assignment-compatible with a variable if
\begin{enumerate}
\item
The type of the value is the same as the type of the variable,
\item
The type of the value is an instance of a subclass for the class associated
with the variable,
\item
The value is the polymorphic constant {\tt NIL},
\item
Both the variable and value are function types, and
\begin{enumerate}
\item
The number of arguments and their storage form match
\item
The type associated with each by-value or by-name parameter in the value is
type-assignment with the equivalent types in the target variable,
\item
The type associated with each by-reference parameter is the same in both
cases (a test which can be implemented by determining that the values are
type-assignable in both directions), and
\item
The type associated with the returned value is type-assignable to the type of
the target.
\end{enumerate}
\end{enumerate}

No automatic conversions are done by Leda,\footnote{There is one small
exception to the rule that Leda performs no automatic conversions.
Relational values will be automatically converted into a boolean value
when they are used in the test expression in an {\tt if} or {\tt while}
statement.} not even the conversion of integer to real.
All such changes must be explicitly specified by the user.
(Mixed mode arithmetic operations are implemented by defining
several forms of the arithmetic operators at the global level).

Parameters which are passed by-name cannot be used as the target of
an assignment, nor can identifiers which are declared as referring to
constant values.
Parameters which are passed by-value are treated
the same as local variables which have been initialized with the value
of the actual argument.  Note that this is a form of assignment,
and thus the comments relative to the semantics of assignment (above)
are true for by-value parameters as well.  This is occasionally a source
of confusion, when an operation that changes the state of a formal parameter
value will be observed to have also changed the state of the actual
parameter value used in a function call.

Return statements are not permitted in the body defined at the global
level.  A return statement with a value can only be used in an
function which returns a value of a type to which the expression
can be assigned.  Conversely, if a return statement is used within
a function that does not produce a value, then no expression
can be used with the return statement.

Return statements are not permitted within the statement associated
with the relational version of the {\tt for} statement.

The expression used to control execution in an {\tt if} statement or
a {\tt while} loop must be either boolean type, or relational type
(which will be converted into boolean type).

An {\tt else} clause is matched with the closest surrounding {\tt if} statement.

The first expression used in the first and second form of the {\tt for}
statement must be of type relation.  In the second form of the {\tt for}
statement the second expression must be type boolean.
In the form of the {\tt for} statement using assignment (the third
form of {\tt for} statement), the reference and all expressions
must be type integer.

Parenthesis must be used in a procedure call, even if no actual
arguments are being passed.

The keyword {\bf cfunction} introduces a run-time system call; an invocation
of an underlying operation that is normally not type-checked.
It is the responsibility of the programmer to ensure that no
type errors can be introduced through the use of this mechanism.
Normally calls on {\bf cfunctions} are hidden within other
functions.  Implementations may make further restrictions on
the use of this mechanism.  (For example, in the Leda Interpreter there
is a fixed set of cfunctions, which can only be modified by recompiling
the interpreter).

In a procedure call the actual arguments being passed must match in number and
compatibility the arguments given in the declaration of the target function.

Definitions:
expression, functionCall : Section~\ref{gr7},
reference: Section~\ref{gr8}.

\section{Expressions}\label{gr7}

\begin{tabular}{l r l}
expression & ::= & andExpression \\
& $\mid$ & expression {\em orSymbol} andExpression \\ \\
andExpression & ::= & notExpression \\
& $\mid$ & andExpression {\em andSymbol} notExpression \\ \\
notExpression & ::= & relationalExpression \\
& $\mid$ & \verb+~+ notExpression \\
& $\mid$ & reference {\bf is} type \\
& $\mid$ & reference {\bf is} type {\bf (} identifierList {\bf )} \\ \\
relationalExpression & ::= & plusExpression \\
& $\mid$ & plusExpression {\em relationalOperator} plusExpression \\ \\
plusExpression & ::= & timesExpression \\
& $\mid$ & plusExpression {\em plusOperator} timesExpression \\ \\
timesExpression & ::= & functionCall \\
& $\mid$ & timesExpression {\em timesOperator} functionCall \\
& $\mid$ & {\em plusOperator} functionCall \\ \\
functionCall & ::= & basicExpression \\
& $\mid$ & functionCall {\bf (} optionalExpressionList {\bf )} \\
& $\mid$ & {\bf cfunction} {\em identifier} {\bf (} optionalExpressionList {\bf )} $->$ type \\ \\
\end{tabular}

The {\em or} symbol is \verb+|+.
The {\em and} symbol is \verb+&+.
The six relational operators are $<$, $<=$, $=$, $<>$, $>=$ and $>$,
The two plus operators are $+$ and $-$.
The three times operators are $*$, $/$ and \verb+%+, the latter denoting
remainder.
The {\bf cfunction} keyword is described in Section~\ref{gr6}.

The {\bf is} keyword introduces a type pattern matching operation.
The expression returns a boolean true value if the expression to the
left of the {\bf is} is an instance or subclass of the type represented
by the identifier to the right of the keyword.  The identifiers in
the optional list following the test must match the data fields
defined for the type.  If the identifier list is present and the test
is successful, the data fields are copied out of the value into the
variables, in effect undoing the actions of the constructor for the class.

In a function call the actual arguments being passed must match in number and
compatibility the arguments given in the declaration of the target function.
Arguments passed to by-reference parameters need not be references; if
non-reference a temporary variable will be generated, the value of
the parameter will be assigned to the temporary, and the reference passed
to the function will be that of the temporary.

Definitions:
type: Section~\ref{gr3},
optionalExpressionList: Section~\ref{gr6},
basicExpression: Section~\ref{gr8}.

\section{Basic Expression}\label{gr8}

\begin{tabular}{l r l}
basicExpression & ::= & reference \\
& $\mid$ & {\em constant} \\
& $\mid$ & {\bf (} expression {\bf )} \\
& $\mid$ & basicExpression {\bf [} typeList {\bf ]} \\
& $\mid$ & {\bf [} expressionList {\bf ]} \\
& $\mid$ & {\bf function} valueArguments optionalReturnType {\bf ;} declarations body \\ \\
reference & ::= & {\em identifier} \\
& $\mid$ & functionCall {\bf .} {\em identifier} \\ \\
\end{tabular}

Constants can be either string constants, integer constants, or
real (floating-point) constants (see Section~\ref{gr11}).

A reference must denote a declared identifier (which can be either
constant, variable, type or function).  In the first
form the identifier must be accessible in the scope containing the reference,
while in the second form the identifier must be accessible in the scope
denoted by the function call, which must be a class type.

A square bracket surrounding a type-list following a basic expression
is used to describe the resolution of qualified types in a function expression.

Square brackets surrounding an expression list is used to define
an array literal.  All the expressions appearing in the list must have
the same type.

Definitions:
declarations, body: Section~\ref{gr1},
typeList: Section~\ref{gr3},
valueArguments, optionalReturnType: Section~\ref{gr4},
expression, expressionList, functionCall: Section~\ref{gr7}.

\section{Comments and Whitespace}\label{gr9}

Any text between matching curly braces is treated as a comment and
will be ignored by the parser.  This is the only legal use for curly
brace characters.

Outside of literal strings, space, tab and newline characters have no
meaning.

\section{Identifiers and Keywords}\label{gr10}

Identifiers must begin with a letter or underscore, and consist of
an arbitrary number of letters, digits, or underscore characters.

In addition to the textual names for operator symbols (see Section~\ref{gr12}),
the following tokens are reserved as keywords, and cannot
be used to define identifiers:

\begin{center}
\begin{tabular}{l l l l l}
begin & byName & byRef & cfunction & class \\
const & defined & do & else & end  \\
for & function & if & include & of \\
return & then & to & type & var \\
while & is \\
\end{tabular}
\end{center}

\section{Textual names for Operator Symbols}\label{gr12}

The meaning of unary and binary operators in any given context is provided
by the implementation of a function with the {\em textual name\/} for the
operator symbol.  The following table gives the various operators
and their associated textual name.

\begin{center}
\begin{tabular}{| l | l |}
\multicolumn{2}{c}{\em binary operators} \\
\hline
symbol & name \\
\hline
\verb-+- & plus \\
\verb+-+ & minus \\
\verb+*+ & times \\
\verb+/+ & divide \\
\verb+%+ & remainder \\
\verb+&+ & and \\
\verb+|+ & or \\
\verb+<+ & less \\
\verb+<=+ & lessEqual \\
\verb+>+ & greater \\
\verb+>=+ & greaterEqual \\
\verb+==+ & sameAs \\
\verb+~=+ & notSameAs \\
\verb+=+ & equals \\
\verb+<>+ & notEquals \\
\hline
\end{tabular} \begin{tabular}{| l | l |}
\multicolumn{2}{c}{\em unary operators} \\
\hline
symbol & name \\
\hline
\verb+~+ & not \\
\verb+-+ & negation \\
\hline
\end{tabular}
\end{center}

Functions that define operator symbols are the only function names that
can be overloaded (be associated with more than one function body defined
in the same scope).  Such overloading can only take place at the
global scope.

\section{Constants}\label{gr11}

There are three types of constants in Leda.  These are string constants,
integer constants, and real (or floating-point) constants.

String constants are surrounded by a matching pair of double-quote
marks.  String constants cannot span multiple input lines, however
the following escape conventions can be used to represent special
characters:

\begin{center}
\begin{tabular}{| l | l |}
\hline
{\em sequence} & {\em meaning} \\
\hline
\verb+\n+ & {\em newline} \\
\verb+\t+ & {\em tab} \\
\verb+\b+ & {\em backspace} \\
\verb+\\+ & {\em backslash} \\
\verb+\"+ & {\em quote mark} \\
\hline
\end{tabular}
\end{center}

There is no separate type for character constants.
Strings of length 1 can be used in place of character values.

Integer constants consist of a sequence of digit characters.
The underlying architecture may impose restrictions on the size
of integer constants that can be recognized.

A real (or floating-point) constant consists of a non-empty sequence
of digit characters, followed by a fractional part and/or an
exponent part.  A fractional part consists of a decimal point
(period) followed by a non-empty sequence of digit characters.
An exponent part consists of the literal character {\bf E} followed
by an optional sign and a non-empty sequence of digit characters.
